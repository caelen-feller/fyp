\PassOptionsToPackage{force}{filehook}
\documentclass[envcountsect notes]{beamer}       % print frame + notes
%\documentclass[notes=only]{beamer}   % only notes
%\documentclass{beamer}              % only frames

\mode<presentation>
{
  \usetheme{Darmstadt}      
  \usecolortheme{rose} 
  \usefonttheme{professionalfonts}
  \setbeamertemplate{navigation symbols}{}
  \setbeamertemplate{caption}[numbered]
  \setbeamertemplate{headline}{}
} 
\usepackage{xpatch}
\usepackage{graphicx}
\usepackage{xcolor}
\usepackage{amsmath}
\usepackage{bookmark}
\usepackage{caption}
\usepackage{svg}
\usepackage[english]{babel}
\usepackage[utf8]{inputenc}
\graphicspath{ {./images/} }

\newcommand{\bb}[1]{\mathbb{#1}}
\newcommand{\abs}[1]{\lvert #1 \rvert}
\newcommand{\norm}[1]{\left\lVert#1\right\rVert}
\newcommand{\hol}[1]{\mathcal{O}\left(#1\right)}
\newtheorem{proposition}{Proposition}

\title[Solving Laplace Problems with Corner Singularities]{Solving Laplace Problems with Corner Singularities}
\author[Feller,Caelen]{Caelen Feller \\ \vspace{1cm} \footnotesize Supervised by: Prof. Kirk M. Soodhalter}
\date{January 24 2020}

\xpatchcmd{\itemize}
  {\def\makelabel}
  {\setlength{\itemsep}{0.7em}\def\makelabel}
  {}
  {}

\begin{document}

\frame{\titlepage}

%%%%%%%%%%%%%%%%%%%%%%%%%%%%%%%%%%%%%%%%%%%%%%%%%%%%%%%%%%%%%%%%
%%%%%%%%%%%%%%%             INTRODUCTION         %%%%%%%%%%%%%%%
%%%%%%%%%%%%%%%%%%%%%%%%%%%%%%%%%%%%%%%%%%%%%%%%%%%%%%%%%%%%%%%%

\section{Introduction}
\subsection{Overview}
\begin{frame}{Theory Overview}
    \begin{columns}
        \column{0.45\textwidth}
        \begin{block}{Approximation Scheme}
            \begin{gather*}
                r(z) = \sum^{N_1}_{j=1} \underbrace{\frac{a_j}{z - z_j}}_{\text{Newman}} 
                \\+ \sum^{N_2}_{j=0}\underbrace{b_j(z-z_*)^j}_{\text{Runge}}
            \end{gather*}
        \end{block}
        \column{0.5\textwidth}
        \begin{figure}[t]
            \includegraphics[trim={11cm 5cm 10cm 5cm},clip,width=0.45\paperwidth]{L_poles.pdf}
        \end{figure}
    \end{columns}
\end{frame}

\subsection{Background}
\begin{frame}{Background}
    \begin{block}{A short timeline of results used}
    \begin{itemize}
        \item 1885 - Carl Runge proves Runge's theorem, allowing uniform convergence using polynomials within the bulk.
        \item 1964 - D.J. Newman proves root-exponential convergence for $\lvert x \rvert$ using rational approximations.
        \item 2003 - Herbert Stahl proves root-exp for $\lvert x \rvert ^\alpha$
        \item 2019 - Trefethen et. al show root-exp. convergence for this.
    \end{itemize}
    \end{block}
    \begin{definition}[Root-Exponential Convergence]
        Given a sequence $\left\{x_n\right\}$, converging to a true value $x$.
        Convergence of the sequence is root-exponential iff 
        $$\epsilon_n = \lvert x_n - x \rvert = O\left(\exp\left(-C\sqrt{N}\right)\right),\ N \rightarrow \infty,\ C > 0$$ 
    \end{definition}   
\end{frame}

\subsection{Goals}
\begin{frame}{Goals}
    \setbeamerfont*{itemize/enumerate body}{size=\large}
    \setbeamerfont*{itemize/enumerate subbody}{parent=itemize/enumerate body}
    \setbeamerfont*{itemize/enumerate subsubbody}{parent=itemize/enumerate body}
        \begin{itemize}
            \item Investigation and implementation of Trefethen's results.
            \item Extension of theory and/or implementation completed for: 
           \begin{itemize}
                \item Non-continuous boundary conditions.
                \item Non-Dirichlet boundary conditions.
                \item Curved domains. 
            \end{itemize}
            \item Verification of Trefethen's results vs traditional methods
        \end{itemize} 
\end{frame}

%%%%%%%%%%%%%%%%%%%%%%%%%%%%%%%%%%%%%%%%%%%%%%%%%%%%%%%%%%%%%%%%
%%%%%%%%%%%%%%%         THEORY OVERVIEW          %%%%%%%%%%%%%%%
%%%%%%%%%%%%%%%%%%%%%%%%%%%%%%%%%%%%%%%%%%%%%%%%%%%%%%%%%%%%%%%%

\section{Theory}

\subsection{Theorems}
\begin{frame}{Root-Convergence Theorem}
    \begin{block}{Theorem}
        Let $\Omega$ be a convex polygonal domain with corners $\omega_1\dots,\omega_m$, 
        $\ f \in\hol{\Omega}$, with holom. continuation to $\Delta_{\epsilon_k}(\omega_k) \setminus $ the exterior bisector of domain at $\omega_k$.
        \begin{gather*}
            \text{If }\exists\ \delta > 0 \text{ s.t. }\lvert f(z) - f(\omega_k) \rvert  = O(\lvert z - \omega_k\rvert^\delta),\ z \rightarrow \omega_k \implies \\ 
            \exists\text{ degree }n\text{ rational functions }\left\{ r_i \right\},\ i > 0 \text{ s.t. }
            \lvert f - r\rvert_\Omega = \\
            O(\exp(-C\sqrt{n})), C > 0,\ n \rightarrow \infty \text{, with the finite poles of each } \\
            r_n \text{ clustered exponentially along the exterior bisectors of }\\
            \text{the domain such that the number of poles near a corner grows }\\
            \text{in proportion to } n,\text{as } n \rightarrow \infty
        \end{gather*}
        
    \end{block}
\end{frame}

% \begin{frame}{Non-Convex Domains}
%     Can position along both sides for all $\theta < \pi$. 
%     Potential Function for general non-convex domains.
% \end{frame}

%%%%%%%%%%%%%%%%%%%%%%%%%%%%%%%%%%%%%%%%%%%%%%%%%%%%%%%%%%%%%%%%
%%%%%%%%%%%%%%%         IMPLEMENTATION           %%%%%%%%%%%%%%%
%%%%%%%%%%%%%%%%%%%%%%%%%%%%%%%%%%%%%%%%%%%%%%%%%%%%%%%%%%%%%%%%
\section{Implementation}

\subsection{Algorithm}

\begin{frame}{Algorithm}
    For boundary $\Gamma$, with corners $\omega_1, \dots, \omega_m$, boundary function 
    $h$ and error $\epsilon$, with increasing values of $n$. 
    \begin{enumerate}
        \item Fix $N_1=O(mn)$ poles clustered outside corners.
        \item Fix $N_2+1=O(n)$ monomials, $1, \dots, (z-z_*)^{N_2}$, and set $N = N_1 + N_2 + 1$
        \item $M \approx 3N$ sample points on the boundary clustered near the corners.
        \item Evaluate at sample points to form $M \times N$ matrix A and $M$-vector b
        \item \textbf{Solve the least-squares problem}, $Ac \approx b$ for c.
        \item Exit loop if $\norm{Ax - b}_\infty < \epsilon$.
    \end{enumerate}
\end{frame}

\begin{frame}{Boundary Sampling Method}
    \begin{figure}[t]
        \includegraphics[trim={10cm 6cm 10cm 6cm},clip, width=0.7\textwidth]{L_poles_bdd.pdf}
        \vspace{-3em}
        \caption*{Boundary Sampling points in green}
    \end{figure}
\end{frame}

\begin{frame}{Full Solution}
    \begin{figure}[t]
        \includegraphics[width=\textwidth]{L_result.pdf}
        \vspace{-3em}
        \caption*{L-Shaped domain with solution}
    \end{figure}
\end{frame}

\begin{frame}{Demonstration}
    \begin{figure}[t]
        \includegraphics[width=\textwidth]{L_discont.pdf}
        \vspace{-3em}
        \caption*{Discontinuous Boundary Conditions}
    \end{figure}
\end{frame}

\begin{frame}{Demonstration}
    \begin{figure}[t]
        \includegraphics[width=\textwidth]{L_neumann.pdf}
        \vspace{-3em}
        \caption*{Robin Boundary Conditions}
    \end{figure}
\end{frame}

\begin{frame}{Demonstration}
    \begin{figure}[t]
        \includegraphics[width=\textwidth]{circular_abs_lim.pdf}
        \vspace{-3em}
        \caption*{Curved boundaries}
    \end{figure}
\end{frame}

\subsection{Conditioning and Stability}
\begin{frame}{Limit to Accuracy}
    \begin{figure}[t]
        \vspace{-1em}
        \includegraphics[width=\textwidth]{L_abs_limit.pdf}
        \vspace{-3em}
        \caption*{Upper bound to accuracy ~ $1e^{-10}$}
    \end{figure}
\end{frame}

%%%%%%%%%%%%%%%%%%%%%%%%%%%%%%%%%%%%%%%%%%%%%%%%%%%%%%%%%%%%%%%%
%%%%%%%%%%%%%%%         CONCLUSION               %%%%%%%%%%%%%%%
%%%%%%%%%%%%%%%%%%%%%%%%%%%%%%%%%%%%%%%%%%%%%%%%%%%%%%%%%%%%%%%%

\section{Conclusion}
\begin{frame}{Further Goals}
    In the remainder of the time allocated for this project, I wish to extend this method to cover 
    at least the following cases:
    \begin{itemize}
        \item Elongated domains
        \item Domains with slits \& multiply connected domains
        \item Non-convex domains \& more complex curved domains
        \item Transmission problems
    \end{itemize}
    \vspace{1em}
    I wish to address the lack of stability in the solution for high precision, whether is a software or analytic issue, and understand where this cap in accuracy comes from.
\end{frame}

\frame[plain]{\centering\Huge Thank you!}

\begin{frame}{Elongated Domain}
    \begin{figure}[t]
        \vspace{-1em}
        \includegraphics[width=\textwidth]{Long_no_fix.pdf}
        \vspace{-3em}
        \caption*{No convergence in case of elongated domain}
    \end{figure}
\end{frame}
\begin{frame}{Elongated Domain}
    \begin{figure}[t]
        \vspace{-1em}
        \includegraphics[width=\textwidth]{Long_fix.pdf}
        \vspace{-3em}
        \caption*{Potential fix to domain}
    \end{figure}
\end{frame}
\begin{frame}{Re-Entrant Spike}
    \begin{figure}[t]
        \vspace{-1em}
        \includegraphics[width=\textwidth]{spike_ok.pdf}
        \vspace{-3em}
        \caption*{Re-Entrant Spike converges slowly}
    \end{figure}
\end{frame}
\begin{frame}{Salient Spike}
    \begin{figure}[t]
        \vspace{-1em}
        \includegraphics[width=\textwidth]{spike_salient.pdf}
        \vspace{-3em}
        \caption*{Salient Spike converges faster}
    \end{figure}
\end{frame}

% (Consider dropping)
\begin{frame}{Harmonic Functions }
    \begin{block}{Harmonic Functions}
        \begin{itemize}
            \item Harmonic functions can be associated with holomorphic ones using Hilbert transform.
            \item Imaginary part of transformed function can also be extended across boundaries and about slits with correct behaviour.
            \item Previous theorem thus holds for harmonic functions
            \item Laplace solutions are harmonic
        \end{itemize}
    \end{block}
\end{frame}

\end{document}
\begin{filecontents*}{references.bib}
\end{filecontents*} 
\bibliography{references}